% Created 2021-02-26 Fri 16:30
% Intended LaTeX compiler: pdflatex
\documentclass[english]{article}
\usepackage[T1, T2A]{fontenc}
\usepackage[lutf8]{luainputenc}
\usepackage[english, russian]{babel}
\usepackage{minted}
\usepackage{graphicx}
\usepackage{longtable}
\usepackage{hyperref}
\usepackage{xcolor}
\usepackage{natbib}
\usepackage{amssymb}
\usepackage{amsmath}
\usepackage{caption}
\usepackage{mathtools}
\usepackage{amsthm}
\usepackage{tikz}
\usepackage{grffile}
\usepackage{extarrows}
\usepackage{wrapfig}
\usepackage{rotating}
\usepackage{placeins}
\usepackage[normalem]{ulem}
\usepackage{amsmath}
\usepackage{textcomp}
\usepackage{capt-of}

\usepackage{geometry}
\geometry{a4paper,left=2.5cm,top=2cm,right=2.5cm,bottom=2cm,marginparsep=7pt, marginparwidth=.6in}

 \usepackage{hyperref}
 \hypersetup{
     colorlinks=true,
     linkcolor=blue,
     filecolor=orange,
     citecolor=black,      
     urlcolor=cyan,
     }

\usetikzlibrary{decorations.markings}
\usetikzlibrary{cd}
\usetikzlibrary{patterns}

\newcommand\addtag{\refstepcounter{equation}\tag{\theequation}}
\newcommand{\eqrefoffset}[1]{\addtocounter{equation}{-#1}(\arabic{equation}\addtocounter{equation}{#1})}


\newcommand{\R}{\mathbb{R}}
\renewcommand{\C}{\mathbb{C}}
\newcommand{\N}{\mathbb{N}}
\newcommand{\rank}{\text{rank}}
\newcommand{\const}{\text{const}}
\newcommand{\grad}{\text{grad}}

\theoremstyle{plain}
\newtheorem{axiom}{Аксиома}
\newtheorem{lemma}{Лемма}
\newtheorem{manuallemmainner}{Лемма}
\newenvironment{manuallemma}[1]{%
  \renewcommand\themanuallemmainner{#1}%
  \manuallemmainner
}{\endmanuallemmainner}

\theoremstyle{remark}
\newtheorem*{remark}{Примечание}
\newtheorem*{solution}{Решение}
\newtheorem{corollary}{Следствие}[theorem]
\newtheorem*{examp}{Пример}
\newtheorem*{observation}{Наблюдение}

\theoremstyle{definition}
\newtheorem{task}{Задача}
\newtheorem{theorem}{Теорема}[section]
\newtheorem*{definition}{Определение}
\newtheorem*{symb}{Обозначение}
\newtheorem{manualtheoreminner}{Теорема}
\newenvironment{manualtheorem}[1]{%
  \renewcommand\themanualtheoreminner{#1}%
  \manualtheoreminner
}{\endmanualtheoreminner}
\captionsetup{justification=centering,margin=2cm}
\newenvironment{colored}[1]{\color{#1}}{}

\tikzset{->-/.style={decoration={
  markings,
  mark=at position .5 with {\arrow{>}}},postaction={decorate}}}
\makeatletter
\newcommand*{\relrelbarsep}{.386ex}
\newcommand*{\relrelbar}{%
  \mathrel{%
    \mathpalette\@relrelbar\relrelbarsep
  }%
}
\newcommand*{\@relrelbar}[2]{%
  \raise#2\hbox to 0pt{$\m@th#1\relbar$\hss}%
  \lower#2\hbox{$\m@th#1\relbar$}%
}
\providecommand*{\rightrightarrowsfill@}{%
  \arrowfill@\relrelbar\relrelbar\rightrightarrows
}
\providecommand*{\leftleftarrowsfill@}{%
  \arrowfill@\leftleftarrows\relrelbar\relrelbar
}
\providecommand*{\xrightrightarrows}[2][]{%
  \ext@arrow 0359\rightrightarrowsfill@{#1}{#2}%
}
\providecommand*{\xleftleftarrows}[2][]{%
  \ext@arrow 3095\leftleftarrowsfill@{#1}{#2}%
}
\makeatother
\author{iliayar Yaroshevskiy Ilya}
\date{\today}
\title{Org mode}
\hypersetup{
 pdfauthor={iliayar Yaroshevskiy Ilya},
 pdftitle={Org mode},
 pdfkeywords={},
 pdfsubject={},
 pdfcreator={Emacs 28.0.50 (Org mode )}, 
 pdflang={English}}
\begin{document}

\maketitle
\tableofcontents


\section{Code example}
\label{sec:org3584a2e}
\begin{minted}[frame=lines,linenos=true,mathescape]{c++}
#include <iostream>

int main() {
	std::cout << "Hello World!" << std::endl; //      (sc)
	return 0
}
\end{minted}

Tes reference sc

Inline code: \texttt{cmake ..}

Using codeblocks result:

\begin{minted}[frame=lines,linenos=true,mathescape]{python}
return "Test"
\end{minted}

\begin{minted}[frame=lines,linenos=true,mathescape]{python}
return "Wow it's: {}".format(s)
\end{minted}

\section{\(\LaTeX\) examples}
\label{sec:orgf8b2a76}
\[ e^{ix} = \cos x + i \sin x \]
\[ \begin{vmatrix} 1 & 1 \\ 1 & 0 \end{vmatrix}^2 = \begin{vmatrix} 2 & 1 \\ 1 & 1 \end{vmatrix} \]

Test latex classes: \(\R\)

\begin{theorem}
Some cool theorem
\end{theorem}
\begin{proof}
Some cool proof of some cool theorem
\end{proof}
\begin{theorem}
Another cool theorem
\end{theorem}
\section{Quote example}
\label{sec:org74b5007}
\begin{quote}
Great clouds overhead
Tiny black birds rise and fall
Snow covers Emacs

---AlexSchroeder
\end{quote}
\section{Dot graph}
\label{sec:org3aeb960}
\begin{minted}[frame=lines,linenos=true,mathescape]{dot}
digraph {
    a -> b [color=blue, label="test"];
    b -> c;
    b -> d;
    d -> e;
    d -> f;
}
\end{minted}
\section{Gnuplot example}
\label{sec:org411bb36}

\begin{minted}[frame=lines,linenos=true,mathescape]{gnuplot}
reset

set title "Putting it All Together"

set xlabel "X"
set xrange [-8:8]
set xtics -8,2,8


set ylabel "Y"
set yrange [-20:70]
set ytics -20,10,70

f(x) = x**2
g(x) = x**3
h(x) = 10*sqrt(abs(x))

plot f(x) w lp lw 1, g(x) w p lw 2, h(x) w l lw 3
\end{minted}
\section{Python example}
\label{sec:orgf859d7e}

\begin{minted}[frame=lines,linenos=true,mathescape]{python}
import matplotlib
matplotlib.use('Agg')
import matplotlib.pyplot as plt
filename = '13_2.png'
plt.plot([0, 1, 1, 2], [1, 1, 0, 0])
plt.ylabel("$hash(s[i : k]) = hash(j : k)$")
plt.xlabel("$s_i$")
plt.yticks([0, 1])
plt.xticks([0, 1, 2])
plt.savefig(filename)
return filename
\end{minted}
\section{Python \& Gnuplot example}
\label{sec:org040dd5d}

\begin{minted}[frame=lines,linenos=true,mathescape]{python}
import math
import numpy as np
y = lambda x: x**2
X = list(np.arange(-10, 10, 0.25))
Y = []
for x in X:
    Y += [y(x)]
return list(zip(X, Y))
\end{minted}

\begin{minted}[frame=lines,linenos=true,mathescape]{gnuplot}
plot data
\end{minted}

\begin{minted}[frame=lines,linenos=true,mathescape]{python}
return data[1]
\end{minted}
\end{document}
